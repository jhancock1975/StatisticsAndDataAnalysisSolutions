% tikz in document class for perceptron picture
\documentclass[12pt, tikz, letterpaper]{article}

\usepackage[utf8]{inputenc} %document endcoding
\title{Statistics and Data Analysis Exercises 1.2.1}
\author{John Hancock}

% for picture of perceptron
\usepackage{tikz}
\usetikzlibrary{matrix,chains,positioning,decorations.pathreplacing,arrows}

\tikzset{basic/.style={draw,fill=blue!20,text width=1em,text badly centered}}
\tikzset{input/.style={basic,circle}}
\tikzset{weights/.style={basic,rectangle}}
\tikzset{functions/.style={basic,circle,fill=blue!10}}

% for note under figures
\usepackage[capposition=top]{floatrow}


% for block quote
\usepackage{etoolbox}

% for cases environment
\usepackage{amsmath}

% for links
\usepackage{hyperref}

% for source code listings
\usepackage{listings}
\lstset
{ %Formatting for code in appendix
    language=R,
    basicstyle=\footnotesize,
    numbers=left,
    stepnumber=1,
    showstringspaces=false,
    tabsize=1,
    breaklines=true,
    breakatwhitespace=false,
}

% for left-justified only
%\usepackage[document]{ragged2e}
% the directive below will indent paragraphs
% but it messes up the perceptron diagram
%\setlength{\RaggedRightParindent}{\parindent}

% one inch margins
%://tex.stackexchange.com/questions/20965/one-inch-margins-and-the-geometry-package
\setlength\topmargin{0pt}
\addtolength\topmargin{-\headheight}
\addtolength\topmargin{-\headsep}
\setlength\oddsidemargin{0pt}
\setlength\textwidth{\paperwidth}
\addtolength\textwidth{-2in}
\setlength\textheight{\paperheight}
\addtolength\textheight{-2in}

\setlength{\parindent}{10ex} % Default is 15pt.

\usepackage{setspace}
\doublespacing

% for keeping tables in line with text
\usepackage{float}
\restylefloat{table}

\begin{document}
\maketitle
%\RaggedRight
\section*{Introduction}

This document holds some solutions, and references we used in determining the answers to the questions in
\cite{statsAndData}, exercises in section 1.2.1 of the same book.

\section{Exercise 1}
\subsection{Part2} we use \cite{stackHistogram} to generate the histogram in the ipython notebook.

% we format references according to:
% https://ieee-dataport.org/sites/default/files/analysis/27/IEEE%20Citation%20Guidelines.pdf
\begin{thebibliography}{99}


\bibitem{statsAndData} A.Abebe, J. Daniels, and J. W. McKean, Statistics and Data Analysis,
  2nd ed. Western Michigan University,Kalamzoo, MI: Statistical Computation Lab (SCL), 2001. [E-book] Available:
  \url{http://www.stat.wmich.edu/s160/hcopy/book.pdf}.

\bibitem{stackHistogram} StackOverflow user Alexander, August 2017. Available: StackOverflow, https://stackoverflow.com/posts/31029857/revisions . [Accessed December 27, 2018].
\end{thebibliography}


\end{document}
